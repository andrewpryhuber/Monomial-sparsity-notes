\documentclass[10pt,letterpaper,oneside]{amsart}
\usepackage[square,sort,comma,numbers]{natbib}


\usepackage{hyperref}
\usepackage{cleveref}
\usepackage[letterpaper]{geometry}
\usepackage[draft,inline,nomargin]{fixme}
\usepackage{graphicx}
\fxsetup{theme=color}

% Definitions 1st Compiled 9/30/16

%packages for more functionality
\usepackage{fullpage} %to get full 8.5-by-11 inch margins
\usepackage{bbm} %more extensive blackboard bold alphabet
\usepackage{amssymb} %gives you all the symbols you could want
\usepackage{graphicx} %helpful for putting in pictures
\usepackage{enumerate} %control list environments precisely
\usepackage{mathabx} %more symbols
\usepackage{xcolor} %colored text
\usepackage{soul} %for striking out
%\usepackage{tikz} % com
\usepackage{tikz-cd} % commutative diagrams
\usepackage{algorithm}
\usepackage[noend]{algpseudocode}

%bold letters

\let \P \relax
\let \H \relax

\newcommand{\A}{\mathbb{A}}
\newcommand{\P}{\mathbb{P}}
\newcommand{\H}{\mathbb{H}}
\newcommand{\Z}{\mathbb{Z}}
\newcommand{\N}{\mathbb{N}}
\newcommand{\Q}{\mathbb{Q}}
\newcommand{\R}{\mathbb{R}}
\newcommand{\C}{\mathbb{C}}
\newcommand{\F}{\mathbb{F}}
\newcommand{\B}{\mathbb{B}}
\newcommand{\T}{\mathbb{T}}
%\renewcommand{\S}{\mathbb{S}} % don't use if want to use the section symbol
\newcommand{\M}{\mathbb{M}}



% cal letters
%\let \AA \relax

%\newcommand{\AA}{\mathcal{A}}
\newcommand{\CC}{\mathcal{C}}
\newcommand{\EE}{\mathcal{E}}
\newcommand{\FF}{\mathcal{F}}
\newcommand{\LL}{\mathcal{L}}
\newcommand{\DD}{\mathcal{D}}
\newcommand{\RR}{\mathcal{R}}
\newcommand{\TT}{\mathcal{T}}
\newcommand{\PP}{\mathcal{P}}
\newcommand{\MM}{\mathcal{M}}
\newcommand{\HH}{\mathcal{H}}
\newcommand{\GG}{\mathcal{G}}
\newcommand{\NN}{\mathcal{N}}
\newcommand{\II}{\mathcal{I}}
\newcommand{\VV}{\mathcal{V}}

% shorter greek

%for theta
\let \th \relax
\let \o \relax

\newcommand{\al}{\alpha}
\newcommand{\be}{\beta}
\newcommand{\ga}{\gamma}
\newcommand{\de}{\delta}
\newcommand{\De}{\Delta}
\newcommand{\la}{\lambda}
\newcommand{\La}{\Lambda}
\newcommand{\ka}{\kappa}
\newcommand{\Si}{\Sigma}
\newcommand{\si}{\sigma}
\newcommand{\ep}{\varepsilon}
\newcommand{\th}{\theta}
\newcommand{\Ga}{\Gamma}
\newcommand{\om}{\omega}
\newcommand{\Om}{\Omega}

\newcommand{\grad}{\nabla}
\newcommand{\hess}{\nabla^2}


% because I am overcome by laziness
\newcommand{\g}{\nabla}
\newcommand{\o}{\circ} % useful for polars and compositions


% random commands

\newcommand{\1}{\mathbbm{1}}
\newcommand{\mb}[1]{\vec{\mathbf{#1}}}
\newcommand{\mbk}{\color{red}}
\newcommand{\Tr}{\text{tr}\,}
\newcommand{\cross}{\times}
\newcommand{\mc}{\mathcal}
\newcommand{\mf}{\mathfrak}
\renewcommand{\d}{\partial}
\renewcommand{\mc}{\mathcal}
\newcommand{\w}{\wedge}
\newcommand{\minus}{\smallsetminus}
\renewcommand{\bar}{\overline}
\renewcommand{\tilde}{\widetilde}
\renewcommand{\phi}{\varphi}




%inner products
\let \< \relax
\let \> \relax
\newcommand{\<}{\langle}
\newcommand{\>}{\rangle}


% Matrix norm with three bars
\newcommand{\mn}[1]{{\left\vert\kern-0.25ex\left\vert\kern-0.25ex\left\vert #1 \right\vert\kern-0.25ex\right\vert\kern-0.25ex\right\vert}}

% For ultimate laziness
\newcommand{\BM}{\begin{matrix}}
\newcommand{\EM}{\end{matrix}}

%\newcommand{\BA}{ \begin{align*} }     not sure why this doesn't work
%\newcommand{\EA}{ \end{align*} }



% Coordinate vector fields

\newcommand{\ddt}{\frac{\partial}{\partial t}}
\newcommand{\dds}{\frac{\partial}{\partial s}}
\newcommand{\ddx}{\frac{\partial}{\partial x}}
\newcommand{\ddy}{\frac{\partial}{\partial y}}
\newcommand{\ddz}{\frac{\partial}{\partial z}}

\newcommand{\ddxi}{\frac{\partial}{\partial x^i}}
\newcommand{\ddxj}{\frac{\partial}{\partial x^j}}

\newcommand{\ddsi}{\frac{\partial}{\partial s^i}}
\newcommand{\ddsj}{\frac{\partial}{\partial s^j}}

\newcommand{\ddxx}[1]{\frac{\partial}{\partial x^{#1}} }
\newcommand{\ddss}[1]{\frac{\partial}{\partial s^{#1}} }





%I want to write certain math expressions in Roman font instead of the italicized math font.  Especially functions, but things like sine and cosine have built in \sin and \cos commands.  \DeclareMathOperator lets you make similar function expressions of your own.

\let \Im \relax
\let \Re \relax
\let \div \relax
\let \O \relax
\let \Tr \relax
\let \null \relax


%\DeclareMathOperator{\grad}{grad} % the nabla symbol is better
\DeclareMathOperator{\GL}{GL}
\DeclareMathOperator{\SL}{SL}
\DeclareMathOperator{\SO}{SO}
\DeclareMathOperator{\O}{O}
\DeclareMathOperator{\Span}{Span}
\DeclareMathOperator{\Aut}{Aut}
\DeclareMathOperator{\Gal}{Gal}
\DeclareMathOperator{\im}{im}
\DeclareMathOperator{\ch}{ch}
\DeclareMathOperator{\ord}{ord}
\DeclareMathOperator{\stab}{stab}
\DeclareMathOperator{\cl}{cl}
\DeclareMathOperator{\Log}{Log}
\DeclareMathOperator{\Ker}{Ker}
\DeclareMathOperator{\lcm}{lcm}
\DeclareMathOperator{\cov}{Cov}
\DeclareMathOperator{\Int}{Int}
\DeclareMathOperator{\id}{id}
\DeclareMathOperator{\Orb}{Orb}
\DeclareMathOperator{\rank}{rank}
\DeclareMathOperator{\nullity}{nullity}
\DeclareMathOperator{\Id}{Id}
\DeclareMathOperator{\Im}{Im}
\DeclareMathOperator{\Re}{Re}
\DeclareMathOperator{\supp}{supp}
\DeclareMathOperator{\sgn}{sgn}
\DeclareMathOperator{\Lie}{Lie}
\DeclareMathOperator{\curl}{curl}
\DeclareMathOperator{\Vol}{Vol}
\DeclareMathOperator{\diag}{diag}
\DeclareMathOperator{\div}{div}
\DeclareMathOperator{\ad}{ad}
\DeclareMathOperator{\Ad}{Ad}
\DeclareMathOperator{\range}{Range}
\DeclareMathOperator{\Range}{Range}
\DeclareMathOperator{\kernel}{Ker}
\DeclareMathOperator{\tr}{Tr}
\DeclareMathOperator{\Tr}{Tr}
\DeclareMathOperator{\adj}{adj}
\DeclareMathOperator{\null}{Null}
\DeclareMathOperator{\Null}{Null}
\DeclareMathOperator{\proj}{proj}
\DeclareMathOperator{\epi}{epi}
\DeclareMathOperator{\dom}{dom}
\DeclareMathOperator{\conv}{conv}
%\DeclareMathOperator{\cl}{cl} % apparently this is built in
\DeclareMathOperator{\aff}{aff}
\DeclareMathOperator{\dist}{dist}
\DeclareMathOperator{\argmax}{argmax}
\DeclareMathOperator{\argmin}{argmin}
\DeclareMathOperator{\ri}{ri}
\DeclareMathOperator{\rb}{rb}
\DeclareMathOperator{\co}{co}
\DeclareMathOperator{\cone}{cone}
\DeclareMathOperator{\lev}{lev}
\DeclareMathOperator{\Ab}{Ab}
\DeclareMathOperator{\ddiag}{ddiag}
\DeclareMathOperator{\deh}{deh}
\DeclareMathOperator{\rowspan}{rowspan}
%\DeclareMathOperator{\hom}{hom}


%Spacing for theorems, lemmas, etc...

\newtheorem{theorem}{Theorem}[section]
\newtheorem{definition}[theorem]{Definition}
\newtheorem{proposition}[theorem]{Proposition}
\newtheorem{lemma}[theorem]{Lemma}
\newtheorem{example}[theorem]{Example}
\newtheorem{corollary}[theorem]{Corollary}
\newtheorem{question}[theorem]{Question}
\newtheorem{conjecture}[theorem]{Conjecture}
\newtheorem{remark}[theorem]{Remarks}


% spacing for function restriction
\newcommand\restrict[2]{\ensuremath{\left.#1\right|_{#2}}}


\newcommand{\etal}{\textit{et al}.~}
\newcommand{\ie}{\textit{i}.\textit{e}.~}
\newcommand{\eg}{\textit{e}.\textit{g}.~}

\DeclareMathOperator{\Syz}{Syz}
\DeclareMathOperator{\row}{row}

\renewcommand{\bf}{\mathbf}

\title{Macaulay matrix row selection}
\author{Andrew Pryhuber}
\date{\today}


\begin{document}

\maketitle


\section{Notation}

\begin{itemize}
    \item Fix polynomial ring $R = \C[x_1, \dots, x_m]$
    \item Macaulay matrix of $I = \< f_1, \dots, f_n\>$ of degree $d$ is $M(d)$.
    \item Syzygy module of $f_1, \dots, f_n$ is $\Syz(f_1, \dots, f_n) = \{\bf{h} : \bf{h}^\top \bf{f} = \sum_{i= 1}^n h_if_i = 0\}$.
    \item $S(f_i,f_j)$ is the s-polynomial of $f_i$, $f_j$.
    \item Given some polynomial $p \in \C[x_1, \dots, x_m]$, for any $d \ge \deg(p)$, $p$ can be identified with some element of the vector space $\C[x_1, \dots, x_m]_d$ consisting of polynomials of degree at most $d$. When we wish to specify in which degree we are considering $p$, we write $p_d$. For a fixed basis of $\C[x_1, \dots, x_m]_d$, $p_d$ is represented by a column vector of size ${ m +d \choose d}$. 
\end{itemize}



\begin{example}
\label{ex:syzreduce}
Let $R = \C[x_1,x_2,x_3]$ and consider the ideal $I = \< f_1,f_2,f_3\> $ where
\begin{align*}
f_1 &= {x}_{1}^{3}-{x}_{1}+{x}_{2}+{x}_{3}-10\\
f_2 &= {x}_{2}^{3}+{x}_{1}-{x}_{2}+{x}_{3}-10\\
f_3 &= {x}_{3}^{3}+{x}_{1}+{x}_{2}-{x}_{3}-10.
\end{align*}
Define $p = S(f_1, f_2) + S(f_2,f_3) = -{x}_{1}^{4}+{x}_{1}^{3}{x}_{2}-2\,{x}_{1}{x}_{2}^{3}-{x}_{1}^{3}{x}_{3
      }+2\,{x}_{2}^{3}{x}_{3}+{x}_{1}{x}_{3}^{3}-{x}_{2}{x}_{3}^{3}+{x}_{3}^{4
      }+10\,{x}_{1}^{3}-10\,{x}_{3}^{3}$. This can certainly be written as a linear combination of the four monomial multiples of the original generators $\{x_2^3 f_1, x_1^3 f_2, x_3^3 f_2, x_2^3 f_3\}$, hence $p_6$ only requires four rows to be represented as an element of the row space of $M(6)$. One can check (using syzReductionEx.m) that $p_4$ is also in the row span of $M(4)$, however $M(4)$ is full rank, so there a unique $h$ such that $p_4 = h^\top M(4)$. This $h$ has 10 non zero entries, i.e., it requires 10 monomials multiples of the generators $f_i$ to produce $p_4$. Increasing degree from 4 to 6 allows us to represent $p$ using fewer monomial multiples. 
\end{example}
Assuming we have chosen a sufficiently large $d$ such that $p_d \in \row M(d)$, these minimal representations of $p_d$ can be found by solving the following minimization problem.
\begin{align}
    &\min \quad \|h\|_0 \\
         &p_d = h^\top M(d). 
\end{align}
Observe that the number of rows of $M(d)$ grows as a polynomial as $d$ increases. Therefore this problem is not tractable as we increase $d$. Instead, we consider the convex relaxation 
\begin{align}
    &\min \quad\|h\|_1 \\
         &p_d = h^\top M(d),
\end{align}
which is the well known basis pursuit problem (see Section 6.5.4 in \cite{boyd2004convex}). While the 1-norm problem is a relaxation, it is common that it yields a sparse solution. For the ideal $I$ and polynomial $p$ from Example \ref{ex:syzreduce}, solving this problem for degrees $d = 4, \dots, 10$ with CVX gives the following results
\begin{table}[]
\begin{tabular}{lll}
$d$ & $\min\|h\|_1$ & numNonZero(h)  \\
4     & 30   & 10 \\
5     & 30   & 10 \\
6     & 4    & 4  \\
7     & 3.8  & 20 \\
8     & 3.5  & 32 \\
9     & 2.7  & 12 \\
10    & 2.67 & 18
\end{tabular}
\end{table}

\newpage


The results beg the question of whether there is a degree $d^*$ specified by a set of generators $f_1, \dots, f_n$ such that a representation of $p_{d^*}$ using the fewest number of rows from $M(d^*)$ also yields a minimal representation of $p_d$ using the rows of $M(d)$ for all $d \ge d^*$. If there is such a finite $d^*$, it must be lower bounded by the quantity $d_S$ which will define below. First we make precise what we mean by the degree of a syzygy.

\begin{definition}
Suppose $\bf{h} \in \Syz(f_1, \dots, f_n)$. Then $\deg(\bf{h}) = \max_i \deg(h_i f_i)$. 
\end{definition}

It is well known that the elements in the left null space of the Macaulay matrix represent syzygies (\cite{batselier2014null}, \cite{stetter2004numerical}). In order for a syzygy $\bf{h}$ to correspond to something in the left null space of $M(d)$, it is necessary that $\deg(\bf{h}) \le d$. In fact, $\bf{h}$ is a syzygy of degree at most $d$ if and only if there is some corresponding $h \in \ker M(d)^\top$ such that $\bf{h}^\top \bf{f} = h^\top M(d)$. Notice that as we increase $d$, it becomes possible to express more syzygies among the $f_i$ as left null space elements of $M(d)$. 
\begin{definition}
Let $d_S$ be the smallest degree such that elements of $\ker(M(d)^\top)$ can generate $\Syz(f_1, \dots, f_n)$ as an $R$-module. 
\end{definition}

From a Gr\"{o}bner basis of $I = \< f_1, \dots, f_n\>$, one can compute generators of $\Syz(f_1, \dots, f_n)$, say $\bf{h}_1,\dots, \bf{h}_s$. If the maximal degree of the $\bf{h}_i$ is $d_{\max}$, then the elements of $\ker(M(d_{\max})^\top)$ can generate $\Syz(f_1, \dots, f_n)$ by construction. Is there a notion of minimal degree generators of $\Syz(f_1, \dots, f_n)$?

For the polynomials from Example \ref{ex:syzreduce}, the syzygy module can be computed in Macaulay2:
\begin{align}
\Syz(f_1, \dots, f_n) = \left \< \begin{pmatrix}
      {x}_{2}^{3}+{x}_{1}-{x}_{2}+{x}_{3}-10\\-{x}_{1}^{3}+{x}_{1}-{x}_{2}-{x}_{3}+10\\0\end{pmatrix},\begin{pmatrix}
      0\\{x}_{3}^{3}+{x}_{1}+{x}_{2}-{x}_{3}-10\\-{x}_{2}^{3}-{x}_{1}+{x}_{2}-{x}_{3}+10\end{pmatrix},\begin{pmatrix}
      {x}_{3}^{3}+{x}_{1}+{x}_{2}-{x}_{3}-10\\0\\-{x}_{1}^{3}+{x}_{1}-{x}_{2}-{x}_{3}+10\end{pmatrix}\right\>
\end{align}
The maximal degree among the generators is 6. When increasing degree from $M(4)$ to $M(6)$, the minimal representations of $p_4$ and $p_6$ are 
$p_4 = (-x_1 +x_2 -x_3 +10)f_1 + (-2x_1 +2 x_3)f_2 + (x_1 -x_2 + x_3 -10)f_3$ and $p_6 = x_2^3 f_1 + (x_3^3 - x_1^3)f_2 - x_2^3 f_3$. Because we have ``uncovered" the top degree term of the syzygies, we can replace the large number of trailing terms with the smaller number of high degree monomial multiples $\{x_2^3 f_1, x_1^3 f_2, x_3^3 f_2, x_2^3 f_3\}$. Of course, this example was carefully crafted so perhaps there are other polynomials that require $M(d)$ for some large $d$ to get a minimal representation. This just suggests that the search for minimal representations should at least involved degree up to the maximal degree of the generators of $\Syz(f_1, \dots, f_n)$. 



\bibliography{references}{}
\bibliographystyle{siam}


\end{document}
